\documentclass[fleqn]{article}

%% Created with wxMaxima 23.05.1

\setlength{\parskip}{\medskipamount}
\setlength{\parindent}{0pt}
\usepackage{iftex}
\ifPDFTeX
  % PDFLaTeX or LaTeX 
  \usepackage[utf8]{inputenc}
  \usepackage[T1]{fontenc}
  \DeclareUnicodeCharacter{00B5}{\ensuremath{\mu}}
\else
  %  XeLaTeX or LuaLaTeX
  \usepackage{fontspec}
\fi
\usepackage{graphicx}
\usepackage{color}
\usepackage{amsmath}
\usepackage{grffile}
\usepackage{ifthen}
\newsavebox{\picturebox}
\newlength{\pictureboxwidth}
\newlength{\pictureboxheight}
\newcommand{\includeimage}[1]{
    \savebox{\picturebox}{\includegraphics{#1}}
    \settoheight{\pictureboxheight}{\usebox{\picturebox}}
    \settowidth{\pictureboxwidth}{\usebox{\picturebox}}
    \ifthenelse{\lengthtest{\pictureboxwidth > .95\linewidth}}
    {
        \includegraphics[width=.95\linewidth,height=.80\textheight,keepaspectratio]{#1}
    }
    {
        \ifthenelse{\lengthtest{\pictureboxheight>.80\textheight}}
        {
            \includegraphics[width=.95\linewidth,height=.80\textheight,keepaspectratio]{#1}
            
        }
        {
            \includegraphics{#1}
        }
    }
}
\newlength{\thislabelwidth}
\DeclareMathOperator{\abs}{abs}

\definecolor{labelcolor}{RGB}{100,0,0}

\begin{document}


\noindent
%%%%%%%%
%% INPUT:
\begin{minipage}[t]{4.000000em}\color{red}\bfseries
 --\ensuremath{\ensuremath{>}}	
\end{minipage}
\begin{minipage}[t]{\textwidth}\color{blue}
/*\ Definição\ das\ funções\ trigonométricas\ */\ \ \\
f(x):=sin(x)\$\ \ /*\ função\ seno\ */\\
g(x):=cos(x)\$\ \ /*\ função\ cossseno\ */\\
\\
/*\ Configuração\ dos\ calculos\ simbólicos\ (não\ numérico)\ */\\
/*\ Incremento\ de\ radianos\ para\ gerar\ lista\ de\ valores\ para\ “x”\ */\ \\
\ \ \ \ arc:\ \ensuremath{\pi}/12\ \$\\
/*\ Lista\ com\ os\ elementos\ para\ o\ cabeçalho\ de\ colunas\ */\\
cabeçalhoDaTabela:\ ["Arco(rad)",\ arc,\ 2*arc,\ 3*arc,\ 4*arc\ ]\$\\
\ \ \ \ \\
numer:\ true\$\ /*\ a\ partir\ daqui\ o\ cálculo\ será\ numérico.\ */\ \\
\\
arc:\ \ensuremath{\pi}/12\ \$\ \ \ /*\ a\ variável\ agora\ recebe\ um\ valor\ numérico\ */\\
\\
/*\ Cálculo\ usando\ a\ função\ "f(x)"\ \ */\\
/*\ lista\ */\ linha1:\ makelist(f(x),x,\ [arc,2*arc,3*arc,4*arc])\$\\
linha1:\ ["Seno",\ linha1[1],\ linha1[2],\ linha1[3],\ linha1[4]]\$\\
/*\ a\ lista\ "linha1"\ é\ alterada\ para\ incluir\ um\ cabeçalho\ de\ linha\ (à\ esquerda)\ */\\
\\
/*\ Cálculo\ usando\ a\ função\ "g(x)"\ \ */\\
/*\ lista\ */\ \ linha2:\ makelist(g(x),x,\ [arc,2*arc,3*arc,4*arc])\$\\
linha2:\ ["Cosseno",linha2[1],\ linha2[2],\ linha2[3],\ linha2[4]]\$\\
/*\ a\ lista\ "linha2"\ é\ reconstruída\ para\ incluir\ nela\ um\ cabeçalho\ de\ linha\ */\\
\\
/*\ Gerar\ a\ tabela\ dos\ resultados\ */\\
fpprintprec\ :\ 3\$\ \ /*definindo\ a\ precisão\ numérica.\ */\ \ \ \ \ \\
\ \\
table\_form([cabeçalhoDaTabela,\ linha1,\ linha2]);\\
/*\ tabela\ com\ \ "table\_form()"\ */\\
\ \\
matrix(cabeçalhoDaTabela,\ linha1,\ linha2);\ \\
/*\ tabela\ com\ a\ função\ "matrix()"\ */
\end{minipage}

\noindent%

\end{document}
