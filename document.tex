% Options for packages loaded elsewhere
\PassOptionsToPackage{unicode}{hyperref}
\PassOptionsToPackage{hyphens}{url}
%
\documentclass[
]{article}
\usepackage{amsmath,amssymb}
\usepackage{lmodern}
\usepackage{iftex}
\ifPDFTeX
\usepackage[T1]{fontenc}
\usepackage[utf8]{inputenc}
\usepackage{textcomp} % provide euro and other symbols
\else % if luatex or xetex
\usepackage{unicode-math}
\defaultfontfeatures{Scale=MatchLowercase}
\defaultfontfeatures[\rmfamily]{Ligatures=TeX,Scale=1}
\fi
% Use upquote if available, for straight quotes in verbatim environments
\IfFileExists{upquote.sty}{\usepackage{upquote}}{}
\IfFileExists{microtype.sty}{% use microtype if available
	\usepackage[]{microtype}
	\UseMicrotypeSet[protrusion]{basicmath} % disable protrusion for tt fonts
}{}
\makeatletter
\@ifundefined{KOMAClassName}{% if non-KOMA class
	\IfFileExists{parskip.sty}{%
		\usepackage{parskip}
	}{% else
		\setlength{\parindent}{0pt}
		\setlength{\parskip}{6pt plus 2pt minus 1pt}}
}{% if KOMA class
	\KOMAoptions{parskip=half}}
\makeatother
\usepackage{xcolor}
\usepackage{longtable,booktabs,array}
\usepackage{calc} % for calculating minipage widths
% Correct order of tables after \paragraph or \subparagraph
\usepackage{etoolbox}
\makeatletter
\patchcmd\longtable{\par}{\if@noskipsec\mbox{}\fi\par}{}{}
\makeatother
% Allow footnotes in longtable head/foot
\IfFileExists{footnotehyper.sty}{\usepackage{footnotehyper}}{\usepackage{footnote}}
\makesavenoteenv{longtable}
\setlength{\emergencystretch}{3em} % prevent overfull lines
\providecommand{\tightlist}{%
	\setlength{\itemsep}{0pt}\setlength{\parskip}{0pt}}
\setcounter{secnumdepth}{-\maxdimen} % remove section numbering
\ifLuaTeX
\usepackage{selnolig}  % disable illegal ligatures
\fi
\IfFileExists{bookmark.sty}{\usepackage{bookmark}}{\usepackage{hyperref}}
\IfFileExists{xurl.sty}{\usepackage{xurl}}{} % add URL line breaks if available
\urlstyle{same} % disable monospaced font for URLs
\hypersetup{
	pdftitle={catenária estudo final},
	hidelinks,
	pdfcreator={LaTeX via pandoc}}

\title{catenária estudo final}
\author{}
\date{}

\begin{document}
	\maketitle
	
	Simulação onda sobre um fio estendido
	
	\begin{longtable}[]{@{}
			>{\raggedright\arraybackslash}p{(\columnwidth - 2\tabcolsep) * \real{0.5000}}
			>{\raggedright\arraybackslash}p{(\columnwidth - 2\tabcolsep) * \real{0.5000}}@{}}
		\toprule()
		\endhead
		{ -\/-\textgreater{} } & \begin{minipage}[t]{\linewidth}\raggedright
			{ {kill}{(}{all}{)}{\$} {/* limpar a memória*/}{\hfill\break
				}{L}{:} {5}{\$} ~~ {/* comprimento L do fio: */}{\hfill\break
				}{/* */}{\hfill\break
				}{H}{:}{11} {\$} ~ {/* tensão horizontal no ponto crítico (ponto de
					equilíbrio)*/}{\hfill\break
				}{/* */}{\hfill\break
				}{w}{:}{0}{.}{5}{\$} ~ {/* peso untário */}{\hfill\break
				}{/* */}{\hfill\break
				}{y\_0}{:}{2}{\$} ~ {/* ordenada do ponto de sustentação
					*/}{\hfill\break
				}{/* */}{\hfill\break
				}{/* Calcular a curva catenária genérica */}{\hfill\break
				} ~~ {a}{:} {H}{/}{w}{\$} ~~~ {/* diretriz ~*/}{\hfill\break
				} ~~ {/* Determinar ~a abscissa do seu ponto de sustentação
					*/}{\hfill\break
				} ~~~~~~ {numer}{:} {true}{\$}{\hfill\break
				} ~~~~~~ {sol1}{:} {solve}{(}{a}{·}
				{sinh}{(}{x\_1}{/}{a}{)}{=}{L}{/}{2}{,} {x\_1}{)}{\$}{\hfill\break
				} ~~~~~~ {x\_1}{:} {last}{(}{last}{(}{sol1}{)}{)}{\$}{\hfill\break
				} ~~~~~~ {cat1}{(}{x}{)}{:}{=}{a}{·} {cosh}{(}{(}{x}{)}{/}{a}{)}
				{\$}{\hfill\break
				}{/* */}{\hfill\break
				}{/* Ajustar a ordenada para o ponto de sustentação*/}{\hfill\break
				}{/* Transformar a catenária para o primeiro quadrante */}{\hfill\break
				} ~~~~~~ {cat}{(}{x}{)}{:}{=} {a}{·}
				} {/* */}{\hfill\break
				}{/* Gerar gráficos e plotar */} {\hfill\break
				} ~~ {/* */}{\hfill\break
				} ~~~~~~ {wxplot\_size}{:}{{[}}{1900}{,}{400}{{]}}{\$}{\hfill\break
				} ~~~~~~ {wxdraw2d}{(}{\hfill\break
				} ~~~~~~ {title}{=}{"Fio Estendido"}{,}{\hfill\break
				} ~~~~~~
				{xlabel}{=}{"x"}{,}{ylabel}{=}{"y"}{,}{grid}{=}{true}{,}{\hfill\break
				} ~~~~~~
				{color}{=}{blue}{,}{line\_type}{=}{dots}{,}{key}{=}{"Catenária"}{,}{\hfill\break
				} ~~~~~~ {line\_width} {=} {8}{,}{\hfill\break
				} ~~~~~~ {grid}{=}{{[}}{5}{,}{5}{{]}}{,}{\hfill\break
				} ~~~~~~ {nticks}{=}{300}{,}{\hfill\break
				} ~~~~~~ {line\_type} ~~~ {=} {solid}{,}{\hfill\break
				} ~~~~~~ {explicit}{(} {cat}{(}{x}{)} {,}{x}{,} {0}{,}
				{2}{·}{x\_1}{)}{\hfill\break
				}{)}{\$}{\hfill\break
				} ~~~~~~ {/* ~Representar o efeito ondulatório sobre o perfil do fio
					*/}{\hfill\break
				} ~~~~~~~~~~ {t}{:} {1}{\$} {\hfill\break
				} ~~~~~~~~~~ {k}{:} {2}{·}{\%pi}{/}{(}{2}{·}{x\_1}{)}{\$} {\hfill\break
				} ~~~~~~~~~~ {/**/}{\hfill\break
				} ~~~~~~~~~~ {for} {i}{:} {1} {while} {i} {\textless{}}{=} {4} {do} {(}
				~~~~ {/* estrutura de repetição */}{\hfill\break
				} ~~~~~~~~~~~~~~~~~~ {wxdraw2d}{(}{\hfill\break
				} ~~~~~~~~~~~~~~~~~~ {title}{=}
				{concat}{(}{"Instante="}{,}{float}{(}{t}{)}{)} ~ {,}{\hfill\break
				} ~~~~~~~~~~~~~~~~~~
				{xlabel}{=}{"x"}{,}{ylabel}{=}{"y"}{,}{grid}{=}{true}{,}{\hfill\break
				} ~~~~~~~~~~~~~~~~~~
				{color}{=}{blue}{,}{line\_type}{=}{dots}{,}{key}{=}{"Perfil de
					onda"}{,}{\hfill\break
				} ~~~~~~~~~~~~~~~~~~ {line\_width} {=} {8}{,}{\hfill\break
				} ~~~~~~~~~~~~~~~~~~ {grid}{=}{{[}}{5}{,}{5}{{]}}{,}{\hfill\break
				} ~~~~~~~~~~~~~~~~~~ {nticks}{=}{300}{,}{\hfill\break
				} ~~~~~~~~~~~~~~~~~~ {line\_type} ~~~ {=} {dots}{,}{\hfill\break
				} ~~~~~~~~~~~~~~~~~~ {explicit}{(} {cat}{(}{x}{)} {+}
				{0}{.}{1}{·}{cat}{(}{x}{)} {·} {2}{·} {(}{sin} {(}{k}{·}{x}{)}{)}
				{2}{·}{x\_1}{)}{)}{,}{\hfill\break
				} ~~~~~~~~~~~~~~~~~~ {t}{:} {t}{+}{1}{\hfill\break
				} ~~~~~~~~~~ {)}{\$}{\hfill\break

		\bottomrule()
	\end{longtable}
	

	
\end{document}
